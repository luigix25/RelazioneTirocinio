\documentclass[a4paper,11pt]{article}
\usepackage[utf8]{inputenc}
\usepackage[italian]{babel}
\usepackage[maxbibnames=99,backend=bibtex]{biblatex}
\usepackage{hyperref}
\usepackage{listings}
\usepackage{color}

\addbibresource{ref.bib}

% define the title
\author{Luigi Leonardi}
%\pagestyle{headings}

\title{Tirocinio su Posit}
\date{}


\definecolor{mygreen}{rgb}{0,0.6,0}

\lstset{  
	numbers=left,
	numbersep=5pt,
	commentstyle=\color{mygreen},
	stringstyle=\color{red}\ttfamily  
}


%link cliccabili
\hypersetup{colorlinks=true, linktoc=all,  linkcolor=black,citecolor=black}

\begin{document}
	% generates the title
	\maketitle
	% insert the table of contents
	\tableofcontents
	
%	\cite{epfl1}
	

\newpage
	\section{Introduzione}
	\subsection{Posit}

	Il Posit è un formato di numero in virgola mobile ideato da John Gustafson, in alternativa allo standard IEEE 754. L'idea di base è fondamentalmente la stessa, anche nei Posit è presente un bit per il segno, dei bit per l'esponente e dei bit per la mantissa, le principali differenze consistono nella presenza di un "super esponente" o regime e nel non avere un numero di bit fissato per quest'ultimo e per la mantissa. \\

	/* Inserire immagine formato */ \newline
	Il vantaggio nell'avere un super esponente, la cui lunghezza non è definita, permette di ottenere un range di numeri molto più flessibile, il suo contributo è pari a $2^{2^{es}}$ con es il numero di bit dell'esponente\footnote{L'esonente è l'unico campo ad avere una dimensione fissa}, permettendo quindi una riduzione del numero di bit assegnati a quest'ultimo campo oppure un incremento di range.\newline Questi bit di regime non sono altro che una sequenza di cifre binarie identiche, terminate dal complemento di esse. \newline Ad esempio: 0001 rappresenta -3, dove 3 è il numero degli 0 ed 1 è il terminatore.\footnote{Regimi che iniziano per 0 sono negativi, per 1 invece sono positivi. Lo 0 è rappresentato come 10}\cite{epfl1}
	
	\subsection{Note sui Test}
	Tutti i test svolti sui Posit sono stati effettuati sfruttando la libreria in C++ BFP\cite{libbfp} 
	implementando, dove necessario, le funzionalità mancanti.\newline I Posit scelti sono stati a 32/64 bit, con 0 bit di esponente\footnote{Ha senso non utilizzare bit di esponente, in quanto si sfrutta il super esponente}.
%	\subsection{Cippi}
%	\ldots{} ora e' in distilleria.
\newpage
\section {Test}
\subsection{Boh}

\section{Codice}
\subsection{Test Sigmoide}
\begin{lstlisting}[language=C++]
	//Prova.cpp
	printf("Ciao");
	
	
\end{lstlisting}
	


\newpage

	\nocite{articolo1}
	
	\printbibliography[title=Bibliografia]

	

\end{document}